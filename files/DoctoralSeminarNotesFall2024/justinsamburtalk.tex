\documentclass[10pt, oneside]{article} 
\usepackage{amsmath, amsthm, amssymb, calrsfs, wasysym, verbatim, bbm, color, graphics, geometry}


\geometry{tmargin=.75in, bmargin=.75in, lmargin=.75in, rmargin = .75in}  

\newcommand{\R}{\mathbb{R}}
\newcommand{\C}{\mathbb{C}}
\newcommand{\Z}{\mathbb{Z}}
\newcommand{\N}{\mathbb{N}}
\newcommand{\Q}{\mathbb{Q}}
\newcommand{\Cdot}{\boldsymbol{\cdot}}

\newtheorem{thm}{Theorem}
\newtheorem{defn}{Definition}
\newtheorem{conv}{Convention}
\newtheorem{rem}{Remark}
\newtheorem{lem}{Lemma}
\newtheorem{cor}{Corollary}


\title{In Situ Spectroscopy and Nanoscale Imaging of Electrochemical Energy Conversion and Storage Systems}
\author{[Chirayu Salgarkar]}
\date{Fall 2024}

\begin{document}

\maketitle
\tableofcontents

\vspace{.25in}


\subsection{Introduction}
Justin Sambur is Associate Prof in Dept of Chemistry at CSU Fort Collins. He looks like a electrochem guy. Nuren may find this interesting. To show for later. He's a Sloan Research Fellow. 
\section{Big Picture}
He likes to measure things on a microscope. How can you watch a battery particle to an ion insertion reaction? 
MOS2 semiconductor thin af

They use total-internal-reflection-fluorescence. Lego microscope on to beads. 
Breaking the diffraction barrier
\begin{defn}
Pseudocapacitance describes an electrochemical mechanism that appears to be capacitive but in fact originates from charge transfer processes across the electrode/elctrolyte interface.
\end{defn}

Capacitors: Powerdrill. 

Electrochemical double layer capacitors have zero solid-state mass transfer and have no real solid structural rearrangement. However, they have a non-faradaic process at the electrode surface, which means that there is low energy density. 


High-rate electrochemical storage (Augustyn et al) shows how to lithiate niobium oxide, and then figure out in 1 minute, you can charge a solid mass. This really fast capacitive like storage can charge the bulk, not the surface. Interestingly, this is not mass-transport limited. That means this material acts like a pseudocapacitor. 

In the next five years, he wants to improve the mass transport.

There are then controversy onn presence of supercapacitors. 

He developed a single nanoparticle electrooptical imagine solution, which changes the elctronic structure of the materia. Which means the battery material changes color. This allows you to select for the Faradaic process, and allows you to better find the linke of pseudocapacitance. 

He then does ion insertion chemistry, which reports the redox changes, decouples from double-layer charging.

THey then do single particle electro optical imaging, which makes it easier to physically identify the particles necessary for finding the important stuff. 


He figures out the solid-state diffusion of Li-ions through $WO_3$ is surface limiting process. Then, he makes a model for finding storage contributions in $WO_3$ nanoparticles, which quantifies the total fluxes in battery-like and pseudo-like. 

He now defines what ``near surface'' means. Rods now have a gradient of lithiation. 



\subsection{Nanoscale Imaging and spectroscopy of 2D photoelectrodes}

\section{Conclusions}






\end{document}
