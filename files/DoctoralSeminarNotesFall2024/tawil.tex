\documentclass[10pt, oneside]{article} 
\usepackage{amsmath, amsthm, amssymb, calrsfs, wasysym, verbatim, bbm, color, graphics, geometry}


\geometry{tmargin=.75in, bmargin=.75in, lmargin=.75in, rmargin = .75in}  

\newcommand{\R}{\mathbb{R}}
\newcommand{\C}{\mathbb{C}}
\newcommand{\Z}{\mathbb{Z}}
\newcommand{\N}{\mathbb{N}}
\newcommand{\Q}{\mathbb{Q}}
\newcommand{\Cdot}{\boldsymbol{\cdot}}

\newtheorem{thm}{Theorem}
\newtheorem{defn}{Definition}
\newtheorem{conv}{Convention}
\newtheorem{rem}{Remark}
\newtheorem{lem}{Lemma}
\newtheorem{cor}{Corollary}


\title{Kareem Tawil: Peer evaluation}
\author{[Chirayu Salgarkar]}
\date{Fall 2024}

\begin{document}

\maketitle
\tableofcontents

\vspace{.25in}

\section{A critical review of Insights into drop-on-demand metal additive manufacturing through integrated experimental and computation study}

\subsection{Introduction}
Where was the paper published? Additive Manufacturing. They focus on Drop-on-demand Molten Metal Jetting (DOD-MMJ). It's fast, accurate, and safe, but is more than 5 years away from industrial adoption. 

Presenter is clear when analyzing the presentation, and uses hand geestures that clearly show what he's doing.

Really like how they use color to show what they liked and didn't like for analysis of experimental methoology, and the errors in the experimental methodology (i.e surface roughness, single test case, imagery of droplet impact). 

Really clear simulation data. Very impressive how you remade the entire setup and got very close to experimental procedure. 





Could use figure captions to explain what is going on in each step. Also, cite the location of the image. 

I think the overall font and color scheme of the presentatiom makes a few things harder to see. For instance, when describing issues, the red is a little hard to see.

\subsection{Research Goals}
Wanted to find how copper droplets bind onto copper substrates, and how residual stress delaminates such droplets. This is significant becasue they are microdroplets, which is small. Also Cu droplet deposition, which isn't really well-studied - deposition of copper requires FIB-SEM analysis. 

Analysis of Methodology:
There is numerical and experimental testing. One case, one thermal condition. The way they describe the methodology shows thermal modeling, etc. 

Issues: no measurment of surface roughness, one test case, and no imagery of droplet impact. 

Analysis of thermal modeling was also conducted. This has some issues, in even thermal contact resistance. Mechanical modeling notates that mesh and mesh sensitivity analysis seems strong, but issue that there is assumption that droplet attachment of substrate is full. 

\section{Results and discussion}
Notates weird drop morphology, and small ``splat factor'', this apparently undermines this entire research information. Authors argue that there's self-peeling, presenter holds skepticism. Droplet-droplet-bonding argument doesn't make sense for thermal modeling - experiemntal modeling. 

\section{Conclusions}
Experimental data is still very important, but analysis terrible. 





\end{document}
