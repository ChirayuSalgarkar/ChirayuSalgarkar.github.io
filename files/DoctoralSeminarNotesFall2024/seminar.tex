\documentclass[10pt, oneside]{article} 
\usepackage{amsmath, amsthm, amssymb, calrsfs, wasysym, verbatim, bbm, color, graphics, geometry}


\geometry{tmargin=.75in, bmargin=.75in, lmargin=.75in, rmargin = .75in}  

\newcommand{\R}{\mathbb{R}}
\newcommand{\C}{\mathbb{C}}
\newcommand{\Z}{\mathbb{Z}}
\newcommand{\N}{\mathbb{N}}
\newcommand{\Q}{\mathbb{Q}}
\newcommand{\Cdot}{\boldsymbol{\cdot}}

\newtheorem{thm}{Theorem}
\newtheorem{defn}{Definition}
\newtheorem{conv}{Convention}
\newtheorem{rem}{Remark}
\newtheorem{lem}{Lemma}
\newtheorem{cor}{Corollary}


\title{Running Lecture Outline: Doctoral Seminar}
\author{[Chirayu Salgarkar]}
\date{Fall 2024}

\begin{document}

\maketitle
\tableofcontents

\vspace{.25in}

\section{27-AUG-24}

\subsection{Miscellaneous}

This class is in Institute Hall. Tuesdays Thursdays at 12:30 PM

\subsection{Faculty Intro}
Dr. Risa Robinson (MIE). I've met her a couple of times already. 
\subsection{Syllabus}
Syllabus is in myCourses. Class meets only once per week. Thursday time is for special speakers, if applicable. Primarily, meet on Tuesdays. 

Class is generally to provide awareness of research and their societal context. More specifcally, this class asks:

\begin{rem}
    How can you communicate effectively?
\end{rem}
TechComm need not be persuasive. 


\subsection{Scientific vs Engineering Research}
So, what exactly is research?

    
\begin{defn}[Research]
    \textbf{Research} is the intellectual \underline{process} to create new knowledge. 
\end{defn}
This is essentially language building. Measurements prove or disprove our understanding of such theory or data. 
\begin{defn}
Inductive inference moves from observations to a more general conclusion or hypothesis.  -- this is scientific method/research
\end{defn}
\begin{defn}
    Deductive inference is more specific knowledge derived from more general principles.
\end{defn}
On the other hand, engineering research is new knowledge in the form of a new entity of tangible existence. 

So, ultimately, what is scientific method/research?
\begin{defn}(Scientific Method)
    The Scientific method is the principles and procedures for the systematic pursuit of knowledge involving recogntion and formulation of a proven collection of data through observation and experiment, and the formulation and testing of hypothesis. 
\end{defn}
Only a sith (or a mathematician) deals with absolutes. For that reason, we use statistics, because we're not sure. 

Engineering research contrasts from this philosophy because it deems to \textit{create something new.}

\section{03-SEP-24}
\subsection{How to read a paper: the structured abstract}
A research abstract is a concise, standalone summary of your work. It answers the problem by which you are going to solve, as well as methods used, findings, and implications. They are SHORT. 150-250 words, just like in high school. 
Basically, people only really the abstract. This means it's VERY important. It's indexed, searchable in PubMed, Google Scholar, and may be the ONLY part available online. This means that this is a gateway to cited list. 

There are (generally) six sections to an abstract: there's an introduction, objectives, methods, results, conclusion, and impact. Space allocation: Results >> methods == Impact > objectives > intro == conclusion, but really focus on results. This is a general guideline. 
\subsection{Structured vs Traditional abstract}
Structured abstracts have headings, while traditional abstracts don't. To be frank, this is mainly a medical thing. But, they are very useful! It helps a lot to write an abstract as a structured abstract (with headings), which can circulate to co-authors for review and edits. If the jouranl does not need a structured abstract, removing the headings is easy. 
More importantly, this forces you to write normally. This makes things easier to read. 

Always write your paper before you write your abstract. Check journal guidelines for word limits, structure to save time afterward to edit. 


\end{document}
