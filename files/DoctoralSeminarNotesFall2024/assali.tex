\documentclass[10pt, oneside]{article} 
\usepackage{amsmath, amsthm, amssymb, calrsfs, wasysym, verbatim, bbm, color, graphics, geometry}


\geometry{tmargin=.75in, bmargin=.75in, lmargin=.75in, rmargin = .75in}  

\newcommand{\R}{\mathbb{R}}
\newcommand{\C}{\mathbb{C}}
\newcommand{\Z}{\mathbb{Z}}
\newcommand{\N}{\mathbb{N}}
\newcommand{\Q}{\mathbb{Q}}
\newcommand{\Cdot}{\boldsymbol{\cdot}}

\newtheorem{thm}{Theorem}
\newtheorem{defn}{Definition}
\newtheorem{conv}{Convention}
\newtheorem{rem}{Remark}
\newtheorem{lem}{Lemma}
\newtheorem{cor}{Corollary}


\title{Mohammad Assali: Peer evaluation}
\author{[Chirayu Salgarkar]}
\date{Fall 2024}

\begin{document}

\maketitle
\tableofcontents

\vspace{.25in}

\section{Insights into SportVU by STATS Perform}

\subsection{Did well}
Really like Formatting of slides, and also the diagrams makes a lot of sense.

Good volume, clear concluding values. 

Honesty on value of project. 

Repeating question during question time helped increase clarity. 

\subsection{Questions/improvements}
Speaking slightly fast, I understood, but maybe slow down a small bit. Fonts different between Title Headers and Content slides which is a little distracting. 




\subsection{Introduction}
As temperature increase, power output decrease for photovoltaic. This is bad. Solar PV systems only convert visible radiation to head. Cooling methods rn are air and water cooling (active) but power gain is pretty close to passive methods like heat sinks. 

\subsection{What did they make?}
Cooling channel Double sided PV arangedment. It's a back-facing panel to capture ground, front panel for radiation. Cooling channel for temperatrue control. 

CFD analysis done. In the thermal model, does convection, conduction, and thermal radiation. For the solar radiation model, needed to be three-dimensional. Solves incident radiation of $1000$ Water per quare meter from sun. They used Monte Carlo simulation for photon tracking, which is a quasi exact stochastic photon tracking technique. 


\subsection{Results}
Monte Carlo used to solve for solar radiation, with solar load of $1000$, range of tilt angles from $10$ to $80$ degrees, with variable panel distance to avoid shading. Obtained results were approximately constant. 
Creation of a term called \textit{Radiation Quality} which is ratio of visible incident radiation over total incident ration. 
\subsection{General timeline}
Created in $2005$, featured in tradeshop 2 years post, then demoed in $2009$. Installed in all NBA teams by $2013$, and replaced by $2017$.

\subsection{Research Goals}
Wanted to find how copper droplets bind onto copper substrates, and how residual stress delaminates such droplets. This is significant becasue they are microdroplets, which is small. Also Cu droplet deposition, which isn't really well-studied - deposition of copper requires FIB-SEM analysis. 

Analysis of Methodology:
There is numerical and experimental testing. One case, one thermal condition. The way they describe the methodology shows thermal modeling, etc. 

Issues: no measurment of surface roughness, one test case, and no imagery of droplet impact. 

Analysis of thermal modeling was also conducted. This has some issues, in even thermal contact resistance. Mechanical modeling notates that mesh and mesh sensitivity analysis seems strong, but issue that there is assumption that droplet attachment of substrate is full. 

\section{Results and discussion}
Notates weird drop morphology, and small ``splat factor'', this apparently undermines this entire research information. Authors argue that there's self-peeling, presenter holds skepticism. Droplet-droplet-bonding argument doesn't make sense for thermal modeling - experiemntal modeling. 

\section{Conclusions}
Experimental data is still very important, but analysis terrible. 





\end{document}
