\documentclass[10pt, oneside]{article} 
\usepackage{amsmath, amsthm, amssymb, calrsfs, wasysym, verbatim, bbm, color, graphics, geometry}


\geometry{tmargin=.75in, bmargin=.75in, lmargin=.75in, rmargin = .75in}  

\newcommand{\R}{\mathbb{R}}
\newcommand{\C}{\mathbb{C}}
\newcommand{\Z}{\mathbb{Z}}
\newcommand{\N}{\mathbb{N}}
\newcommand{\Q}{\mathbb{Q}}
\newcommand{\Cdot}{\boldsymbol{\cdot}}

\newtheorem{thm}{Theorem}
\newtheorem{defn}{Definition}
\newtheorem{conv}{Convention}
\newtheorem{rem}{Remark}
\newtheorem{lem}{Lemma}
\newtheorem{cor}{Corollary}

\theoremstyle{definition}
\newtheorem{example}{Example}[section]


\title{Running Lecture Outline: 707}
\author{[Chirayu Salgarkar]}
\date{Fall 2024}

\begin{document}

\maketitle
\tableofcontents

\vspace{.25in}
\section{26-AUG-24}
\subsection{Order, Linear, and PDE vs ODE}
\section{28-AUG-24}

\subsection{Miscellaneous}
I'm doing this to not fall asleep in class. Prof is Phillip Hutton. There are several ways to take the quiz. Quiz, office hours, etc. you can also take it in class. All quiz get one page cheat sheet. 

\subsection{More miscellany}
Prof is doing some incredible projector gymnastics. He should record balance beam events.
\subsection{Verifying solutions using initial conditions}
To verify potential solutions, plug into the original diffeq. Use algebra (ha!) to make LHS = RHS. On the other hand, we can simply plug in various numbers for $x$ and check for equivalency. Obviously, check for domains. 
\begin{example}
   \[ y' = xy^{\frac{1}{2}}\]
Potential Solution: \[y = \frac{1}{16}x^4\]
Then, 
    \[\frac{dy}{dx} = \frac{1}{4}x^3 \]
Then, plug into the original diffeq. We have:
    \[ \frac{1}{4}x^3 = x \dot (\frac{1}{16}x^4)^{\frac{1}{2}}\]
or that
    \[\frac{1}{4}x^3 = \frac{1}{4}x^3 \]
as desired. 
\end{example}

\begin{example}
    \[ (y-x)\frac{dy}{dx} = y-x+8\]
     Potential Solution 1: \[y = 2x + 4\sqrt{x+2}\]
    Potential Solution 2: \[y = x + 4\sqrt{x+2}\]
    Case 1:
     \[\frac{dy}{dx} = 2 + 2(x+2)^{\frac{-1}{2}} \]
     Then, plug into the original diffeq. We have:
     \[ (2x + 4\sqrt{x+2} - x)(2 + 2(x+2)^{\frac{-1}{2}}) = 2x + 4\sqrt{x+2} -x+8\]
     Simplifying,
    \[ (x + 4\sqrt{x+2})(2 + 2(x+2)^{\frac{-1}{2}}) = x+8 + 4\sqrt{x+2}\]
    Consider the case that $x=0$. Then,
    \[ (4\sqrt{2})(2 + 2(2)^{\frac{-1}{2}}) = 8 + 4\sqrt{2}\]
    \[8\sqrt{2} + 8 = 8 + 4\sqrt{2} \] 
    or that 
    \[8\sqrt{2} = 4\sqrt{2}\]
 which is clearly false. 

 Solution 2 works. You plug it in like above, but end with a true statement. 
 \end{example}
 \subsection{IVPs}
 Solving a diffeq yields a \textit{general} solution with unknowns. Using initial values we can then solve for said unknowns. For $n$ unknowns, we need $n$ initial values. 

 Let's do an example!
 \begin{example}
    $y' = y$, where $y(0) = 3$.
    We know our general solution is 
    \[y = Ce^x\]
    but what is $C$?
    clearly, since $y(0) = 3$, and at $x = 0$, $y=c$, $3 = C$. Thus, the equation is really 
    \[y = 3e^x\]
 \end{example}
 \begin{example}
    $y' + 2xy^2 = 0$, where $y(0)=1$.
    The general solution to this is $y = \frac{1}{x+C}$. Plugging in at $x=0$, we have $C=-1$. Final solution is $y = \frac{1}{x-1}$.
 \end{example}
 \begin{example}
    $x'' + 16x = 0$, where $x(\frac{\pi}{2})= -2$, $x'(\frac{\pi}{2}) = 1$. The general solution is 
    \[x = C_1\cos{4t}+C_2\sin{4t}\]

    You plug in twice, you get $C_1 = -2$ and after the second step you get $C_2 = \frac{1}{4}$. Then plug in to general equation. Yay.
\end{example}
\section{30-AUG-24}
\subsection{Integrating Factor}
This is the most fun McIlwain review. When do we use this?
\begin{thm}
    If you can write a differential equation to be of the form $\frac{dy}{dx} + p(x)y = f(x)$, you are eligible to use Integrating Factor.
\end{thm}

The algorithm for solving goes something like this:
\[
\frac{dy}{dx} + p(x)y = f(x)
\]
Multiply both sides by:
\[
I_f = e^{\int{p(x) \, dx}}
\]
You get:
\[
e^{\int{p(x) \, dx}} \frac{dy}{dx} + e^{\int{p(x) \, dx}} p(x) y = e^{\int{p(x) \, dx}} f(x)
\]
Using reverse chain rule:
\[
\frac{d}{dx}[e^{\int{p(x) \, dx}}y]= e^{\int{p(x) \, dx}} f(x)
\]
Integrating both sides, we get:
\[
    e^{\int{p(x) \, dx}}y = \int{e^{\int{p(x) \, dx}} f(x)}
\]

This gets us:
\[y(x) = \frac{\int{e^{\int{p(x) \, dx}} f(x)}}{e^{\int{p(x) \, dx}}}\]

\begin{example}
    \[\frac{dy}{dx} = 5y\]
This seems separable, and it is. But if you were to use integrating factor, it goes like this:
\[
    \frac{dy}{dx} - 5y = 0
\]
Note that this makes $p(t) = -5$
\[
y = Ce^{5x}
\]
\end{example}

\begin{example}
    \[\frac{dy}{dx} + y = e^{3x}\]

Note that this makes $p(x) = 1$, $f(x) = e^{3x}$
\[
e^{x}\frac{dy}{dx} = e^{3x}e^x = e^{4x}
\]
\[
e^{x}y = \frac{1}{4}e^{4x} + C
\]

\[
y = \frac{\frac{1}{4}e^{4x} + C}{e^{x}}
\]
or:
\end{example}




\end{document}
