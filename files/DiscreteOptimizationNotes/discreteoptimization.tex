\documentclass[10pt, oneside]{article} 
\usepackage{amsmath, amsthm, amssymb, calrsfs, wasysym, verbatim, bbm, color, graphics, geometry}

\geometry{tmargin=.75in, bmargin=.75in, lmargin=.75in, rmargin = .75in}  

\newcommand{\R}{\mathbb{R}}
\newcommand{\C}{\mathbb{C}}
\newcommand{\Z}{\mathbb{Z}}
\newcommand{\N}{\mathbb{N}}
\newcommand{\Q}{\mathbb{Q}}
\newcommand{\Cdot}{\boldsymbol{\cdot}}

\newtheorem{thm}{Theorem}
\newtheorem{defn}{Definition}
\newtheorem{conv}{Convention}
\newtheorem{rem}{Remark}
\newtheorem{lem}{Lemma}
\newtheorem{cor}{Corollary}


\title{Running Lecture Outline: Discrete Optimization}
\author{[Chirayu Salgarkar]}
\date{Summer 2024}

\begin{document}

\maketitle
\tableofcontents

\vspace{.25in}

\section{Week 2}

\subsection{Knapsack Problem}

The knapsack problem is a conventional dynamic programming problem. Here's the jist:

Suppose we need to fill a knapsack, with $N$ different desirable items, each with $2$ distinct attributes. 




\end{document}
