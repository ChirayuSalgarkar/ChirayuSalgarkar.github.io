\documentclass[10pt, oneside]{article} 
\usepackage{amsmath, amsthm, amssymb, calrsfs, wasysym, verbatim, bbm, color, graphics, geometry}


\geometry{tmargin=.75in, bmargin=.75in, lmargin=.75in, rmargin = .75in}  

\newcommand{\R}{\mathbb{R}}
\newcommand{\C}{\mathbb{C}}
\newcommand{\Z}{\mathbb{Z}}
\newcommand{\N}{\mathbb{N}}
\newcommand{\Q}{\mathbb{Q}}
\newcommand{\Cdot}{\boldsymbol{\cdot}}

\newtheorem{thm}{Theorem}
\newtheorem{defn}{Definition}
\newtheorem{conv}{Convention}
\newtheorem{rem}{Remark}
\newtheorem{lem}{Lemma}
\newtheorem{cor}{Corollary}


\title{Running Lecture Outline: Interdisciplinary Research Methods}
\author{[Chirayu Salgarkar]}
\date{Fall 2024}

\begin{document}

\maketitle
\tableofcontents

\vspace{.25in}

\section{27-AUG-24}

\subsection{Miscellaneous}

This class is in Gleason, starts at 9:30 AM Tuesdays and Thursdays.


No idea who the prof is. He looks cool though. Makes a parallax error joke with the clock. Quite funny, tbh.


This is a skill-building course. For BME+ChemE, MIE, and ECE. This is mainly for understanding what it takes to understand how to do things for academia. 

\subsection{Faculty Intro}
There's Dr. Vinay Abhyankar (BME), Dr Andres Kwsasinki (ECE), and Dr. Risa Robinson (MIE). 
\subsection{Syllabus}
Syllabus is in myCourses. Highlights include:
\begin{itemize}
    \item Goals are to make them strong Engineering researchers, focus on Interdisciplinary and trans-disciplinary research skills, with \textbf{three themes}: research methods, statistics, and conducting ethical research.
    \item Interdisciplinary STEM research needs communication skills, lit review, and critical evaluation of research, publication process, research proposals and funding, and to prep for quals. 
    \item Research stats involves how to use data well, and analyze data well. This looks like good AP stat. Also how to recognize cherrypicked data. 
\end{itemize}
Grading wise, attendance counts for 10\%, Interdisciplinary Research Methods is 40\%, and Research Stats and Conducting Ethical Research are both 25\%. COME ON TIME. Check myCourses for Assignments. For weeks $1-5$, we have mock quals, that are graded. This will be hard. But it's very valuable. 

There's also the CITI modules stuff. That's the ethics. We have a statistics exam on November 26th. 

\subsection{Scientific vs Engineering Research}
So, what exactly is research?

    
\begin{defn}[Research]
    \textbf{Research} is the intellectual \underline{process} to create new knowledge. 
\end{defn}
This is essentially language building. Measurements prove or disprove our understanding of such theory or data. 
\begin{defn}[Inductive inference]
Inductive inference moves from observations to a more general conclusion or hypothesis.
\end{defn}
This is the scientific method/research.
\begin{defn}[Deductive inference]
    Deductive inference is more specific knowledge derived from more general principles.
\end{defn}
On the other hand, engineering research is new knowledge in the form of a new entity of tangible existence. 

So, ultimately, what is scientific method/research?
\begin{defn}(Scientific Method)
    The Scientific method is the principles and procedures for the systematic pursuit of knowledge involving recogntion and formulation of a proven collection of data through observation and experiment, and the formulation and testing of hypothesis. 
\end{defn}
Only a sith (or a mathematician) deals with absolutes. For that reason, we use statistics, because we're not sure. 

Engineering research contrasts from this philosophy because it deems to \textit{create something new.}

\section{29-AUG-24}
\subsection{Miscellaneous}
Generally, don't use ChatGPT to write a paper, obviously. He doesn't dispute that they are important tools, so the use is fine, but the intellectual generation of content is still your brain. Also, disclose how you are using those tools. If you are using AI to do some code debugging, for instance, please let them know. 

There is now a five minute discussion on the validity of this argument. I am not notating it, because I find it more of a diatribe than actual course content. 
\subsection{Research, Proposals, Funding}
There are three major questions in this lecture:
\begin{rem}[Question]
    Where is research conducted in the US?
\end{rem}
\begin{rem}[Question]
    What is a research grant proposal?
\end{rem}
\begin{rem}[Question]
How are research grants funded?
\end{rem}
Research in the US is in government, academia, and industry. Mainly industry, because of size. 
There are three major Research and development categories: Basic Research, which exists to acquire more research on underlying foundations of phenomena and facts, with no immediate application, applied research, which is primarily towareds a specific, practical aim, and experimental development, which draws on knowledge gained from research, and is directed to producing new products and processes or improving existing products and processes. 

Another way of categorizing this is by the \textit{Technology Readiness Levels}. THey are a set of nine levels that define the level of maturity of the research is. Level 1 is basic research level, whicle Level 9 shows ``Actual system flight proven through successful mission operations", while the other ones are somewhere in between. The full set of levels are in the PPT. It was originally conceptualized by NASA, which explains why it says ``flight". You can make the list analogously based on your own research. 

Governmental research is conducted in a myriad of loci. They include national laboratories, which are ``federally funded research and development centers''. There is also National Institutes, (think NIH, NIST, USDA, DoD, etc.). TLDR: Government is large. 

\subsection{Industrial research}
Industrial research is done by companies. We think big companies (see NVIDIA, Tesla, Google), but small companies do a lot too. Some industries form consortia to collaborate when conducting research, as well as develop roadmaps and standards. This includes semiconductor device fabrication, telecommunication standards, and airlines. 

\subsection{Academic research}
We begin with four questions:
\begin{rem}[Question]
    What are the purpose and goals
\end{rem}
\begin{rem}[Question]
    Is it important or necessary?
\end{rem}
\begin{rem}[Question]
Who does academic research?
\end{rem}
\begin{rem}[Question]
    Who or what funds academic research?
\end{rem}
The primary use of universities is to educated. The goals for PhDs are to educate the next generation of researchers. There are differences in basic, applied, and experimental research. But why are PhDs necessary? 

Professor goes into a diatribe of Bell Labs prof discussing how industry focuses on applied research, and unbalancing the ``three legs'' of research. 

Who does academic research? You, me ``Prof referring to themselves'', postdocs. Who are we funded by? Obviously by university endowments, gifts, foundations, industry contracts, and government contracts. 

Funding varies. There are generally proposals, which vary significantly even within one funding agency, each having many different types of grants ranging from 10k to over 10M, sometimes requiring multiple PIs at multiple institutions. These grant proposals are often reviewed by both peers as well as agency program officers and managers. 


\end{document}
