\documentclass{article}

\usepackage{amsmath, amsthm, amssymb, amsfonts}
\usepackage{thmtools}
\usepackage{graphicx}
\usepackage{setspace}
\usepackage{geometry}
\usepackage{float}
\usepackage{hyperref}
\usepackage[utf8]{inputenc}
\usepackage[english]{babel}
\usepackage{framed}
\usepackage[dvipsnames]{xcolor}
\usepackage{tcolorbox}


\usepackage[font=itshape]{quoting}


\colorlet{LightGray}{White!90!Periwinkle}
\colorlet{LightOrange}{Orange!15}
\colorlet{LightGreen}{Green!15}

\newcommand{\HRule}[1]{\rule{\linewidth}{#1}}

\declaretheoremstyle[name=Theorem,]{thmsty}
\declaretheorem[style=thmsty,numberwithin=section]{theorem}
\tcolorboxenvironment{theorem}{colback=LightGray}

\declaretheoremstyle[name=Proposition,]{prosty}
\declaretheorem[style=prosty,numberlike=theorem]{proposition}
\tcolorboxenvironment{proposition}{colback=LightOrange}

\declaretheoremstyle[name=Principle,]{prcpsty}
\declaretheorem[style=prcpsty,numberlike=theorem]{principle}
\tcolorboxenvironment{principle}{colback=LightGreen}

\declaretheoremstyle[name=Definition,]{defsty}
\declaretheorem[style=defsty,numberlike=theorem]{definition}
\tcolorboxenvironment{definition}{colback=LightLimeGreen}

\setstretch{1.2}
\geometry{
    textheight=9in,
    textwidth=5.5in,
    top=1in,
    headheight=12pt,
    headsep=25pt,
    footskip=30pt
}

% ------------------------------------------------------------------------------

\begin{document}

% ------------------------------------------------------------------------------
% Cover Page and ToC
% ------------------------------------------------------------------------------

\title{ \normalsize \textsc{}
		\\ [2.0cm]
		\HRule{1.5pt} \\
		\LARGE \textbf{\uppercase{Notes for RL}
		\HRule{2.0pt} \\ [0.6cm] \LARGE{RIT, previously Mercer} \vspace*{10\baselineskip}}
		}
\date{}
\author{\textbf{Sutton and Barto} \\
        Compiled by Chirayu Salgarkar\\
		Summer 2024}

\maketitle
\newpage

\tableofcontents
\newpage

% ------------------------------------------------------------------------------
\section{Introduction}
The book is \textit{computational}. All that means is that it cares more about goal-directed learning than other ways, and treats the world ideally. This is pretty cool, I guess. 

\subsection{So, WTF is reinforcement learning?}
In four words, \textit{fuck it we ball}.
\begin{definition}
    RL is learning what to do, or how to map a set of situations to a set of actions, in order to maximize a numerical reward signal.
\end{definition}
Basically, to do RL, you need to have an environment, the agent must be able to understand the state of the environment, and must be able to take actions to affect the state.

How is this different from \textit{supervised learning} or \textit{unsupervised learning}?

Well, supervised learning, is, well, supervised. That means there exist a set of labeled examples used for training provided by a knowledgeable external supervisor. Clearly, the agent has no help and the state of the environment is unadulterated, so it's not supervised. 

How about unsupervised learning? This one is a little more complicated. (It's still easy for anyone with more than two brain cells, mind you. It's just hard for \textit{me}. This is from a NVIDIA blog: 
\begin{quoting}
In unsupervised learning, a deep learning model is handed a dataset without explicit instructions on what to do with it. The training dataset is a collection of examples without a specific desired outcome or correct answer. The neural network then attempts to automatically find structure in the data by extracting useful features and analyzing its structure.
\end{quoting}
This includes stuff like Clustering (think K-Means). However, there is no reward function, so to speak. In RL you're not trying to find hidden structures, you're just trying to maximize a reward signal. So RL isn't unsupervised either. 


The book then talks about stuff in scope, and the tradeoff between exploration and exploitation. For me this is kind of obvious, but I may be wrong. 
\bigbreak

In the mean time, let's play tic-tac-toe. How would you solve it with a value function?
\subsection{Refreshing Mints and Feet}
The first thing you need is a table of numbers, one for each state of the game. The value of that number is the Probability of winning from the state. We can start with all situations where you've lost as $0$, and all situations where you've already won as $1$, and all the other states are defined as $0.5$. Then, we play a bunch of games. Generally, we play greedily, but when we can't, we play randomly, to find \textit{exploratory} moves we may not have had the pleasure to experience otherwise. 

































\begin{proposition}[Question 1]
  A rigid, closed container holds $1.5$ kg of liquid water and $0.1$ kg of water vapor at $350$ kPa.
Calculate the heat transfer to the water required to transform it into a saturated vapor. Also
calculate the volume of the container.  
\end{proposition}
First of all, we draw a picture. 
There are some important words here! Those are \textbf{closed} and \textbf{rigid}.
A closed system has the property that the change of mass is constant. 
\begin{definition}[Closed System]
    For a closed system, $\frac{dm}{dt} = \sum m_{in} - \sum m_{out} = 0$.
\end{definition}

We also have \textbf{rigid}. That means that volume is constant. Obviously.

Now, we plug in the conservation of energy equation.

Then, we are able to simplify, and get something along the form of 
\[(\dot Q - \dot W)dt = \frac{dE}{dt}\]
Integrating, we get:
\[Q - W = E_2 - E_1\]
\[Q - W = m(u_2 - u_1)\]


Note that if pressure and liquid both exist at the same pressure, we are at the iso line, which means we are able to measure the initial quality of the mixture: That can be stated as follows:
\[x_i = \frac{M_g}{M_f + M_g}\]
\[x_i = \frac{0.1 \text{kg}}{(1.5 + 0.1)\text{kg}}\]
\[\frac{1}{16}\]

Now that we have that, we use the numbers in the question. Note that $350 kPa = 3.5 bar$. Using the table, we find that $u_f = 583.95$ and $u_i = 2546.9$ kJ/kg. Then, plugging in the question for the equation of the quality.
\[u_i = u_f + x_i(u_g - u_f)\]
Supposedly, that gives us $706.6$ kJ/kg. 

Now, we do some linear interpolation.
(TODO)
Is there any work in the system?
No, there ain't no stirring.

Then we solve everything and we're done. 



\subsection{Second Law, more formally}
\begin{definition}[Second Law]
The defining equation for entropy change on a differential basis is defined as 
\[dS = \frac{\partial Q}{T}_\text{int, rev}\]  
\end{definition}

In an internally reversible, adiabatic process, entropy is constant. This is called isentropic.

\begin{definition}[adiabatic]
     An \textbf{adiabatic process} is a type of thermodynamic process that occurs without transferring heat or mass between the thermodynamic system and its environment.
\end{definition}


For control bolumes, we get that the temperature at boundary which is the Reservoir, temp must be in absolute units, which is either Kelvin or Rankine.

For a steady prbolem the $\frac{dS_{cv}}{dt}$ is $0$. Note that the rate of entropy production is always positive $\dot \sigma$. For when you do integration shit you get:
\[m_2s_2 - m_1s_1 = \sum_j \frac{Q_j}{T_j} + \sum_t m_es_e + \sigma\]

For internally reversible systems, $Q_{\text{internally reversible}} = \int_1^2 T dS$
\subsection{ideal gas}
For ideal gas system, internal energy and enthalpy are a function of temperature. This is nice. Entropy is not tabulated. In the table, we have $s_0$, noted as "s stupid zero". So, how do you find the entropy change of an ideal gas? 

We start with first law equation.
\[\partial Q - \partial W = dE\]
For an internally reversible system, this can be rewritten as:
\[Tds = Pdv  = du\]
So, 
\[ds - \frac{P}{T}dv = \frac{du}{T}\]
For an ideal gas, 
$Pv = RT$, $du = C_vdT$, $dh = C_pdT$.

So,
\[ds = \frac{du}{T} + \frac{P}{T}dv\]
Rewriting,
\[s_2 - s_1 = C_v \frac{dT}{T} + \frac{R}{V}dV\]
and therefore,
\[s_2 - s_1 = C_v \ln{\frac{T_2}{T_1}} + R \ln{\frac{V_2}{V_1}}\]
And now we can do the stupid $s_0$ thing. 



\section{Lecture 2 bur really it's more problems slay}
Recall that for pure substances, it is very easy to check the table to find the entropy change. For ideal gasses, it's less easy, we have to use the 

\[Tds = Pdv  = du\]
equations, assuming the system is internally reversible. Doing integration shit, we get:
\[s_2 - s_1 = C_v \ln{\frac{T_2}{T_1}} + R \ln{\frac{V_2}{V_1}}\]
Which only works if $C_v$ is constant.

If  $C_v$ is not constant, we use change in enthalpy, which is denoted as:
\[h = u + Pv\]
Then, 
\[dh = du + d(Pv) = du + Pdv - vdP\]
Then, \[Tds - Pdv = dh - Pdv - vdP\]

Evaluating, we get:
\[Tds = dh - vdP\]

For ideal gases, note that:
\[dh = c_pdT\]
Then, 
Evaluating, we get:
\[Tds = c_pdT - vdP\]
and then,
\[ds = c_p\frac{dT}{T} - \frac{v}{T}dP\]

Now, employing the gas law, we know that:
\[Pv = RT\]
Then, it must be true that:
\[\frac{v}{T} = \frac{R}{P}\]

and so,
\[ds = C_p \frac{dT}{T} - \frac{R}{P}dp\]

This is a pain in the ass to integrate. More specifically, $C_p \frac{dT}{T}$ is not a fun way to evaluate. Note that $\int_0^2 C_p \frac{dT}{T}$ is actually $s_0$ evaluated at a point! Very cool. 

\begin{proposition}
    What does the $0$ and $2$ mean here? $0$ is absolute zero, but $2$ is actually the final temperature.
\end{proposition}

\subsection{Irreversiblities}
\begin{definition}
    An irreversibility is defined as the difference in totally reversible work and actual work in the system:
    \[\dot I = \dot W_{rev} - \dot W_{act}\]
\end{definition}
Note that:
\[\dot W_{rev} = \dot W_{act} + \dot {I}\]

This can also be found using entropy production:
\[\dot I = T_0(\dot \sigma_{cv})\]
Here, $T_0$ is lowest natural occurring temperature in the system, or sometimes, $T_{ambient}$. 

\subsection{Heat transfer, work in internally reversible steady state Flow processes. }
Consider this:
\[\dot{m} \left( h_1 + \frac{v_1^2}{2} + gz_1 \right) + \dot{Q} = \dot{m} \left( h_2 + \frac{v_2^2}{2} + gz_2 \right) + \dot{W}
\]

Or neglecting kinetic and potential energy effects:
\[\frac{\dot W_{cv}}{\dot m}_{int rev} = \frac{\dot Q_{cv}}{\dot m}_{int rev} + (h_1 - h_2)\]

If you remember the original equation:
\[dS = \frac{\partial Q}{T}_\text{int, rev}\]  
we can then rewrite this hot garbage with $Tds$ equations.

Then, we get 
\[\frac{\dot W_{cv}}{\dot m}_{int rev} = \int_1^2 {T dS} + (h_1 - h_2) \]
Then, we get, for incompressible fluids:
\[\frac{\dot W_{cv}}{\dot m}_{\text{int rev}} = - \int_1^2 {v dP} \]

For internally reversible work in a pump, 
\[\frac{\dot W_{cv}}{\dot m}_{\text{int rev}} = - \int_1^2 {v dP} \]
Which is:
\[\frac{\dot W_{cv}}{\dot m}_{\text{int rev}} = -{(v)(p_2 - p_1)} \]
where $1$ is inlet and $2$ is exit

\subsection{Adiabatic isentropic garbage}
\[\frac{\dot W_{cv}}{\dot m}_{\text{int rev}} = - \int_1^2 {v dP} \]


\subsection{Example Problem:}
\begin{proposition}[Question 6]
A steady flow, internally reversible compressor is used to compress air isothermally from \( 400 \, \text{kPa} \) to \( 1.8 \, \text{MPa} \). The air enters the compressor at the rate of \( 30 \, \text{m}^3/\text{min} \) at a temperature of \( 35^\circ\text{C} \). The kinetic energy and potential energy of the compressor are negligible. If \( T_0 = 25^\circ\text{C} \), determine:
\begin{enumerate}
    \item The Power input to the compressor (kW or MW)
    \item The entropy produced from the compressor (kW/K)
\end{enumerate}
\end{proposition}

This is basically a plug and chug to ideal gas law, then, we get:
\[\frac{\dot W}{\dot m} = -RT \ln{\frac{P_2}{P_1}} = RT \ln{\frac{P_1}{P_2}}\]

For entropy change, $s_2 - s_1 = -R \ln{{P_2}{P_1}}$

\subsection{Isentropic Turbine Efficiency}
Short digression: turbines are big. 

\[\dot Q_{cv} - \dot W_{cv} + \sum \dot m_i (h_i + E_k + E_p)  \sum \dot m_e (h_c + E_k + E_p) = \frac{dE}{dt}\]
Assuming steady state, $\frac{dE}{dt} = 0$.
If change in $KE$, $PE$ negligible, 
We get:
\[- \dot W_{cv} + \sum \dot m (h_i - h_e) = 0\]
or, 
\[\dot W = \dot m(h_i - h_e)\]
or that
\[\frac{\dot W}{\dot m} = h_1 - h_2\]

\begin{definition}
    The isentropic turbine efficiency is the ratio of actual turbine work to maximum theoretical work, per mass flow unit. This is denoted as:
    \begin{align*}
\eta_{\text{turbine}} &= \frac{h_1 - h_2}{h_1 - h_{2s}} \\[5pt]
\text{where} \quad & \\
\eta_{\text{turbine}} & = \text{Isentropic efficiency of the turbine} \\
h_1 & = \text{Specific enthalpy at the turbine inlet} \\
h_2 & = \text{Actual specific enthalpy at the turbine exit} \\
h_{2s} & = \text{Specific enthalpy at the turbine exit under isentropic conditions}
\end{align*}

\end{definition}

\subsection{Example Problem}
\begin{proposition}
    Water vappor at $5$ bar, $320C$ enters a turbine operating at steady state with volumentric flow rate of $0.65 \frac{m^3}{s}$ and expands adiabatically to an exit state of $1$ bar, $160C$ KE, PE negligible. Find power developed in turbine, in KW, rate of entropy production in KW/K, isentropic turbine efficiency. 
\end{proposition}

We begin with assumptions: Assume Steady state, adiabatic, so $\dot Q = 0$, and that $P_2 = 1 \text{bar}$ and $T_2 = 160C$.

We start with Conservation of energy balance. 
\[\dot Q - \dot W + \sum \dot m_i (h_i + E_k + E_p)  \sum \dot m_e (h_c + E_k + E_p) = \frac{dE}{dt}\]

Using Adiabatic assumption, 

\[- \dot W + \dot m_ih_i - \dotm_eh_e = 0\]


Then, you have conservation of mass balance:
\[\frac{dm}{dt} = \sum \dot m_i - \sum \dot m_e\]
Since mass in inlet and exit is same, 
\[\dot m_i = \dot m_e = \dot m\]

So, 
\[- \dot W + \dot m(h_1 - h_2)  = 0\]
and therefore:
\[\dot W = \dot m(h_1 - h_2)\]

Next, we write the equation of turbine efficiency:
\[
  \begin{align*}
\eta_{\text{turbine}} &= \frac{h_1 - h_2}{h_1 - h_{2s}} \\[5pt]
\text{where} \quad & \\
\eta_{\text{turbine}} & = \text{Isentropic efficiency of the turbine} \\
h_1 & = \text{Specific enthalpy at the turbine inlet} \\
h_2 & = \text{Actual specific enthalpy at the turbine exit} \\
h_{2s} & = \text{Specific enthalpy at the turbine exit under isentropic conditions}
\end{align*}
\]

At $P_1 = 5 \text{bar}$, $T_1 = 320$C. At $v_1 = 0.5416$ m^3/kg, $h_1 = 3105.6$ kJ/kg, $s_1 = 7.5308$ kJ/kgK

Since $\dot m = \frac{V}{\dot v_1}$, plugging into this gets $\dot m = 1.2$kg/sec.

To find $\eta_T$, I need $h_2$. 

















Some times they may find the quality or some other garbage, but you will always have everything to solve the problem.
\section{Linear Systems}

\subsection{Gauss's Method}
Before we begin, some vocabulary would be useful. 
\begin{definition}[linear combination]
A \textbf{linear combination} of $x_1, \ldots, x_n$ has the form:
\[
a_1x_1 + a_2x_2 + a_3x_3 + \cdots + a_nx_n
\]
where the numbers $a_1, \ldots, a_n \in \mathbb{R}$ are the combination's coefficients.
\end{definition}
Then, a linear equation will be of the form $a_1x_2 + a_2x_2 + \dots a_nx_n = d_1$, where $d_1$ is a real-valued constant. A system of linear equation is a bunch of them!

\begin{definition}[System of linear equations]
    A system of $m$ linear equations is of the form:
   \[a_{1,1}x_1 + a_{1,2}x_2 + a_{1,3}x_3 + \dots + a_{1,n}x_n = d_2\]
    \[a_{2,1}x_1 + a_{2,2}x_2 + a_{2,3}x_3 + \dots + a_{2,n}x_n = d_2\]
    \[\dots\]
    \[a_{m,1}x_1 + a_{m,2}x_2 + a_{m,3}x_3 + \dots + a_{m,n}x_n = d_m\]
\end{definition}


\begin{definition}[Solutions to systems of linear equations]
   An \( n \)-tuple (ordered list of $n$ elements) \( (s_1, s_2, \ldots, s_n) \in \mathbb{R}^n \) is a solution of, or satisfies, that equation if substituting the numbers \( s_1, \ldots, s_n \) for the variables gives a true statement: \( a_1s_1 + a_2s_2 + \ldots + a_ns_n = d \) for all equations in the system.. 
\end{definition}
The solution set for a system is the set of all solutions to a system of linear equations.
\bigbreak
How do you find the solution set?
This is a pain to do by brute force, and when you do more complicated systems, you really need a plan. Gauss's method is one such plan. "Turns out there's a rich theory for systems of equations!" - Dr. Troupe.  
\subsection{Examples}
How do you solve this?
\[6x + 5y = -32\]
\[7x + 3y = 5\]

One way we did it was by solving for one variable in one equation, and then to plug it in the other.
For instance, 
\[6x = -32 - 5y \implies x = \frac{-32}{6} - \frac{5}{6}y\]
You then plug into equation 2:
\[7(\frac{-32}{6} - \frac{5}{6}y) + 3y = 5\]
You then solve for $y$, and then solve for $x$ by replacing $y$ with $x$.
\bigbreak
There's another way!
Observe that if $A=B$ and $C=D$, $A+C = B+D$. Also, if $A=B$, for all $x \in \mathbb{R}$, $xA = xB$.
Then, From the system:
\[6x + 5y = -32\]
\[7x + 3y = 5\]
if we were to multiply equation $2$ by $\frac{6}{7}$, we have:
\[6x + 5y = -32\]
\[6x + \frac{18}{7}y = \frac{30}{7}\]

Then, subtracting equation $2$ by equation $1$, we have the form $Ay = B$, where $A, B \in \mathbb{R}$ so $y = \frac{B}{A}$. This is Gauss's Method.
\subsection{Gauss's Method, more officially}
\begin{theorem}[Gauss's Method]
    If a linear system is changed to another by one of these operations:
    \begin{enumerate}
        \item An equation is swapped with another.
        \item An equation has both sides multiplied by a nonzero constant.
        \item An equation is replaced by the sum of itself and a multiple of another.
    \end{enumerate}
    Then the two systems have the same set of solutions.
\end{theorem}
In the future, the $i$-th row of a system will be denoted $\rho_i$. (This is a pretty good pun.)
\bigbreak
Here's an example:

    \[3x_3 = 9\]
    \[x_1 + 5x_2 - 2x_3 = 2\]
    \[\frac{1}{3}x_1 + 2x_2 = 3\]
First, swap $\rho_1$ with $\rho_3$:
    \[\frac{1}{3}x_1 + 2x_2 = 3\]
    \[x_1 + 5x_2 - 2x_3 = 2\]
    \[3x_3 = 9\]
Next, multiply $\rho_1$ by $3$:
    \[x_1 + 6x_2 = 9\]
    \[x_1 + 5x_2 - 2x_3 = 2 \]
    \[3x_3 = 9\]
Now, add $-1\rho_1$ to $\rho_2$:
    \[x_1 + 6x_2 = 9\]
    \[-x_2 - 2x_3 = -7\]
    \[3x_3 = 9\]

Now, we are in row-echelon form. That means we can back-solve and get all the solutions to this system. We get one unique solution. We can prove that this is a unique solution, by using contradiction. 

\begin{definition}[leading variable]
The leading variable is the first variable with a nonzero coeeficient.
\end{definition}

\begin{definition}[echelon form]
A system is in echelon form if each leading variable is to the right of the leading variable in the above row. 
\end{definition}


\newpage

% ------------------------------------------------------------------------------
% Reference and Cited Works
% ------------------------------------------------------------------------------

\bibliographystyle{IEEEtran}
\bibliography{References.bib}

% ------------------------------------------------------------------------------

\end{document}
